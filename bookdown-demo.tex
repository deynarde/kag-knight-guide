\documentclass[]{book}
\usepackage{lmodern}
\usepackage{amssymb,amsmath}
\usepackage{ifxetex,ifluatex}
\usepackage{fixltx2e} % provides \textsubscript
\ifnum 0\ifxetex 1\fi\ifluatex 1\fi=0 % if pdftex
  \usepackage[T1]{fontenc}
  \usepackage[utf8]{inputenc}
\else % if luatex or xelatex
  \ifxetex
    \usepackage{mathspec}
  \else
    \usepackage{fontspec}
  \fi
  \defaultfontfeatures{Ligatures=TeX,Scale=MatchLowercase}
\fi
% use upquote if available, for straight quotes in verbatim environments
\IfFileExists{upquote.sty}{\usepackage{upquote}}{}
% use microtype if available
\IfFileExists{microtype.sty}{%
\usepackage{microtype}
\UseMicrotypeSet[protrusion]{basicmath} % disable protrusion for tt fonts
}{}
\usepackage[margin=1in]{geometry}
\usepackage{hyperref}
\hypersetup{unicode=true,
            pdftitle={The Definite Knight guide},
            pdfborder={0 0 0},
            breaklinks=true}
\urlstyle{same}  % don't use monospace font for urls
\usepackage{natbib}
\bibliographystyle{apalike}
\usepackage{longtable,booktabs}
\usepackage{graphicx,grffile}
\makeatletter
\def\maxwidth{\ifdim\Gin@nat@width>\linewidth\linewidth\else\Gin@nat@width\fi}
\def\maxheight{\ifdim\Gin@nat@height>\textheight\textheight\else\Gin@nat@height\fi}
\makeatother
% Scale images if necessary, so that they will not overflow the page
% margins by default, and it is still possible to overwrite the defaults
% using explicit options in \includegraphics[width, height, ...]{}
\setkeys{Gin}{width=\maxwidth,height=\maxheight,keepaspectratio}
\IfFileExists{parskip.sty}{%
\usepackage{parskip}
}{% else
\setlength{\parindent}{0pt}
\setlength{\parskip}{6pt plus 2pt minus 1pt}
}
\setlength{\emergencystretch}{3em}  % prevent overfull lines
\providecommand{\tightlist}{%
  \setlength{\itemsep}{0pt}\setlength{\parskip}{0pt}}
\setcounter{secnumdepth}{5}
% Redefines (sub)paragraphs to behave more like sections
\ifx\paragraph\undefined\else
\let\oldparagraph\paragraph
\renewcommand{\paragraph}[1]{\oldparagraph{#1}\mbox{}}
\fi
\ifx\subparagraph\undefined\else
\let\oldsubparagraph\subparagraph
\renewcommand{\subparagraph}[1]{\oldsubparagraph{#1}\mbox{}}
\fi

%%% Use protect on footnotes to avoid problems with footnotes in titles
\let\rmarkdownfootnote\footnote%
\def\footnote{\protect\rmarkdownfootnote}

%%% Change title format to be more compact
\usepackage{titling}

% Create subtitle command for use in maketitle
\newcommand{\subtitle}[1]{
  \posttitle{
    \begin{center}\large#1\end{center}
    }
}

\setlength{\droptitle}{-2em}

  \title{The Definite Knight guide}
    \pretitle{\vspace{\droptitle}\centering\huge}
  \posttitle{\par}
    \author{true \\ true}
    \preauthor{\centering\large\emph}
  \postauthor{\par}
      \predate{\centering\large\emph}
  \postdate{\par}
    \date{2019-01-30}

\usepackage{booktabs}
\usepackage{amsthm}
\makeatletter
\def\thm@space@setup{%
  \thm@preskip=8pt plus 2pt minus 4pt
  \thm@postskip=\thm@preskip
}
\makeatother

\begin{document}
\maketitle

{
\setcounter{tocdepth}{1}
\tableofcontents
}
\chapter{Introduction}\label{introduction}

I've read many tutorials on both the game's forums and the Steam
community hub, and 90\% of them, if not more, were either bad or
outdated. The only good one would be the
\href{https://deynarde.github.io/kag-knight-compendium/}{Knight
Compendium}, but it's more a list of tricks than a tutorial (some of the
stuff in here will be taken from it, and I still recommend giving it a
read). Considering we're going to have many new players that will come
with F2P (if any of you are reading this, are from EU and want to
train/1v1 with me - dm bunnie\#8671 or deynarde\#4491 on Discord!), me
and deynarde (the creator of the Compendium) decided to write this
guide.

We hope it will help many of you. We're going to cover everything;
getting better on all levels, playing on CTF, dealing with high ping,
dealing with bugs. It may seem like a lot, but knight combat is very
rich, and if you really want to get better, then this will be perfect
for you.

\chapter{The basics and learning as a
newbie}\label{the-basics-and-learning-as-a-newbie}

tetsetestest

\section{The Knight Class}\label{the-knight-class}

in KAG, the are 3 main classes: the Archer, the Builder and the Knight.
It's pretty much the `main' class of the game; in a good team in CTF
(the main gamemode - capture the flag) or TTH (take the halls) 75\%
players should be playing Knight, and in TDM (team deathmatch) it's the
most played class as well.

As a knight in CTF and TTH, your role will be helping the team advance
forward to get the enemy flag/enemy hall, but before you learn more
about gamemodes and playing in a team, let's talk about the combat
mechanics first.

\chapter{Learning as an average
player}\label{learning-as-an-average-player}

\chapter{Learning as a decent player}\label{learning-as-a-decent-player}

\chapter{Learning as a very good
player}\label{learning-as-a-very-good-player}

\chapter{Being useful in pub CTF}\label{being-useful-in-pub-ctf}

\chapter{Being useful in pub TDM}\label{being-useful-in-pub-tdm}

\chapter{Playing 1v1s, 2v2s}\label{playing-1v1s-2v2s}

\chapter{Playing competitive CTF}\label{playing-competitive-ctf}

\chapter{Dealing with the no animations
bug}\label{dealing-with-the-no-animations-bug}

\chapter{Playing with lag}\label{playing-with-lag}

\bibliography{book.bib,packages.bib}


\end{document}
